\ychapter{Panafricanisme}{
	Ce manuel, à l'usage de la jeunesse africaine se donne pour but de redonner la ferveur de nos héros aux nouvelles générations; la ferveur de Sankara, la tenacité de Lumumba, la détermination de Malcolm X, l'engagement de tous nos héros pour une meileure Afrique, libre, consciente de son histoire et préparant activement son avenir.
	
	A travers citations, biographies, faits historiques, principes sociétaux, et conseils pratiques, ce manifeste participe à l'effort d'éducation et de formation d'une génération d'espoir pour l'Afrique de demain, grande et prospère.
}

	\section{Qu'est-ce le Panafricanisme?}
	Il s'agit du mouvement intellectuel mondial dont le but est de construire, encourager et renforcer l'unité entre tous les afro-déscendants. 
	
	S'inspirant de notre destin historique commun depuis les anciens empires (Nubie, Ghana, Songhai, Egypte, Benin, Kongo,...), 
	les affres de l'esclavage, de la colonisation, des apartheids, en passant par le racisme moderne, 
	le Panafricanisme apporte une vision unique à notre futur à tous: l'UNITE.
	\ypic{nkrumah1}{Kwmamé Mkrumah, premier président du Ghana}
	Les pères des indépendances comme Kwamé Nkrumah du Ghana, Sékou Touré de Guinée, les plus modernes comme Moammar Khadafi, ou encore des afro-descendants comme Malcolm X des Etats-unis et Aimé Césaire des antilles furent parmis les fervents défenseurs de ce mouvement d'unité, primordial à notre essor.
	L'OUA, l'union africaine, est une réalisation de cet effort...

	\clearpage
	\section{Pourquoi le Panafricanisme?}
	Conscient que séparés nous n'arriverons à rien de grand dans les domaines économiques, sociaux et géopolitiques, 
	le panafricanisme promeut une synergie entre les forces sur le continent et dans la diaspora.
	
	Le constat devant une Afrique immensément riche pourtant misérablement pauvre avec une population peu éduquée, reniant sa culture,
	ignorant son histoire et en mal d'identité, reste une réalité consternante qui redonne aujourd'hui vie au panafricanisme.
	
	Conscient que séparés nous n'arriverons à rien de grand dans les domaines économiques, sociaux et géopolitiques, 
	le panafricanisme promeut une synergie entre les forces sur le continent et dans la diaspora.
	\ypic{sankara1}{Thomas Sankara, ancien chef d'état Burkinabé}
	

	Le constat devant une Afrique immensément riche pourtant misérablement pauvre avec une population peu éduquée, reniant sa culture,
	ignorant son histoire et en mal d'identité, reste une réalité consternante qui redonne aujourd'hui vie au panafricanisme.
	
	\section{Quelques célèbres panafricains?}
	\begin{itemize}
		\item Julius Nyéréré
		\item Thomas Sankara
		\item Marcus Garvey
		\item Haile Selasie
		\item W.E.B. Dubois
		\item Patrice Lumumba
		\item Check Anta Diop
		\item Malcolm X
		\item Joseph Ki Zerbo
	\end{itemize}
	\clearpage